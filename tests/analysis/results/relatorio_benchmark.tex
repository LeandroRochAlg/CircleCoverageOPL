\documentclass[12pt,a4paper]{article}
\usepackage[utf8]{inputenc}
\usepackage[brazilian]{babel}
\usepackage{graphicx}
\usepackage{booktabs}
\usepackage{longtable}
\usepackage{float}
\usepackage{geometry}
\usepackage{hyperref}
\usepackage{caption}

\geometry{margin=2.5cm}

\title{Relatório de Benchmark\\Circle Coverage Problem}
\author{Análise Automatizada}
\date{18 de November de 2025}

\begin{document}

\maketitle
\tableofcontents
\newpage

\section{Resumo Executivo}

Este relatório apresenta a análise detalhada dos resultados obtidos no benchmark automatizado 
do problema Circle Coverage. O benchmark foi executado em 73 instâncias 
distintas, testando 6 configurações diferentes, totalizando 788 execuções.

\subsection{Estatísticas Gerais}

\begin{itemize}
    \item \textbf{Total de Instâncias:} 73
    \item \textbf{Total de Execuções:} 788
    \item \textbf{Configurações Testadas:} 6
    \item \textbf{Taxa de Sucesso Global:} 89.0\%
\end{itemize}

\section{Resultados por Configuração}

\subsection{Tabela Resumo}

A Tabela~\ref{tab:resumo} apresenta as estatísticas de desempenho para cada configuração testada.

\begin{table}[H]
\centering
\caption{Resumo de Desempenho por Configuração}
\label{tab:resumo}
\small
\begin{tabular}{lrrrrrr}
\toprule
\textbf{Config} & \textbf{Sucesso} & \textbf{Taxa (\%)} & \textbf{Tempo Méd.} & \textbf{Tempo Med.} & \textbf{Círculos Méd.} & \textbf{Círculos Min-Max} \\
\midrule
Teste1 & 181/183 & 98.9 & 581.06s & 23.75s & 9.5 & 1-24 \\
Teste2 & 152/183 & 83.1 & 37.82s & 11.34s & 8.5 & 1-17 \\
Teste3 & 152/183 & 83.1 & 32.82s & 5.13s & 8.5 & 1-17 \\
Teste4 & 74/80 & 92.5 & 292.08s & 9.39s & 14.1 & 4-26 \\
Teste5 & 77/80 & 96.2 & 413.12s & 10.60s & 15.1 & 4-35 \\
Teste6 & 65/79 & 82.3 & 160.00s & 3.45s & 13.0 & 4-26 \\
\bottomrule
\end{tabular}
\end{table}

\subsection{Análise Gráfica}

\subsubsection{Taxa de Sucesso}

A Figura~\ref{fig:taxa_sucesso} ilustra a taxa de sucesso de cada configuração. 
Configurações com taxa inferior a 100\% indicam dificuldade em resolver todas as instâncias 
dentro do tempo limite estabelecido.

\begin{figure}[H]
\centering
\includegraphics[width=0.8\textwidth]{1_taxa_sucesso.png}
\caption{Taxa de sucesso por configuração}
\label{fig:taxa_sucesso}
\end{figure}

\subsubsection{Tempo de Execução}

A Figura~\ref{fig:tempo} apresenta a análise de tempo de execução. O gráfico à esquerda 
mostra o tempo médio, enquanto o gráfico à direita apresenta a distribuição completa 
dos tempos através de boxplots (escala logarítmica).

\begin{figure}[H]
\centering
\includegraphics[width=\textwidth]{2_tempo_execucao.png}
\caption{Análise de tempo de execução}
\label{fig:tempo}
\end{figure}

\subsubsection{Qualidade da Solução}

A Figura~\ref{fig:qualidade} compara a qualidade das soluções obtidas por cada 
configuração, medida pelo número de círculos necessários. Menores valores indicam 
soluções mais eficientes.

\begin{figure}[H]
\centering
\includegraphics[width=0.9\textwidth]{3_qualidade_solucao.png}
\caption{Qualidade da solução (número de círculos)}
\label{fig:qualidade}
\end{figure}

\section{Análise de Escalabilidade}

A Figura~\ref{fig:escalabilidade} demonstra como o desempenho de cada configuração 
varia em função do tamanho da instância (número de pontos $n$).

\begin{figure}[H]
\centering
\includegraphics[width=\textwidth]{4_escalabilidade.png}
\caption{Escalabilidade: tempo e círculos × tamanho da instância}
\label{fig:escalabilidade}
\end{figure}

\section{Comparação Head-to-Head}

\subsection{Speedup Relativo}

A Figura~\ref{fig:speedup} mostra o speedup de cada configuração em relação ao 
Teste1 (baseline). Valores menores que 1 indicam melhor desempenho.

\begin{figure}[H]
\centering
\includegraphics[width=0.8\textwidth]{5_speedup_relativo.png}
\caption{Speedup relativo ao Teste1 (baseline)}
\label{fig:speedup}
\end{figure}

\subsection{Heatmap de Desempenho}

A Figura~\ref{fig:heatmap} apresenta um heatmap normalizado do tempo de execução 
para as primeiras 30 instâncias. Cores mais escuras (vermelhas) indicam pior desempenho 
relativo naquela instância.

\begin{figure}[H]
\centering
\includegraphics[width=0.9\textwidth]{6_heatmap_desempenho.png}
\caption{Heatmap de desempenho normalizado}
\label{fig:heatmap}
\end{figure}

\section{Análise Estatística}

\subsection{Comparações Pareadas}

A Figura~\ref{fig:comparacoes} apresenta os resultados dos testes estatísticos 
Mann-Whitney U para comparações pareadas entre configurações. Barras vermelhas 
indicam diferenças estatisticamente significativas ($p < 0.05$).

\begin{figure}[H]
\centering
\includegraphics[width=\textwidth]{10_comparacoes_pareadas.png}
\caption{Significância estatística das comparações pareadas}
\label{fig:comparacoes}
\end{figure}

\subsection{Correlação entre Parâmetros}

A Figura~\ref{fig:correlacao} mostra a matriz de correlação de Pearson entre 
os principais parâmetros das instâncias e métricas de desempenho.

\begin{figure}[H]
\centering
\includegraphics[width=0.8\textwidth]{11_correlacao_parametros.png}
\caption{Matriz de correlação entre parâmetros}
\label{fig:correlacao}
\end{figure}

\subsection{Variabilidade entre Repetições}

A Figura~\ref{fig:variabilidade} ilustra o coeficiente de variação (CV) do tempo 
de execução entre repetições da mesma configuração na mesma instância. Valores 
menores indicam maior estabilidade.

\begin{figure}[H]
\centering
\includegraphics[width=\textwidth]{12_variabilidade_repeticoes.png}
\caption{Variabilidade do tempo de execução entre repetições}
\label{fig:variabilidade}
\end{figure}

\section{Análise de Instâncias}

\subsection{Distribuição de Características}

A Figura~\ref{fig:distribuicao} apresenta a distribuição das principais 
características das instâncias testadas.

\begin{figure}[H]
\centering
\includegraphics[width=\textwidth]{13_distribuicao_caracteristicas.png}
\caption{Distribuição das características das instâncias}
\label{fig:distribuicao}
\end{figure}

\subsection{Instâncias Mais Difíceis}

A Figura~\ref{fig:dificeis} identifica as instâncias mais desafiadoras, 
tanto em termos de tempo de execução quanto de taxa de sucesso.

\begin{figure}[H]
\centering
\includegraphics[width=\textwidth]{14_instancias_dificeis.png}
\caption{Instâncias mais difíceis}
\label{fig:dificeis}
\end{figure}

\subsection{Relação Características × Tempo}

A Figura~\ref{fig:carac_tempo} explora a relação entre diferentes características 
das instâncias e o tempo de execução.

\begin{figure}[H]
\centering
\includegraphics[width=\textwidth]{15_caracteristicas_vs_tempo.png}
\caption{Relação entre características e tempo de execução}
\label{fig:carac_tempo}
\end{figure}

\subsection{Perfil de Complexidade}

A Figura~\ref{fig:perfil} classifica as instâncias em categorias de dificuldade 
e analisa as características médias de cada categoria.

\begin{figure}[H]
\centering
\includegraphics[width=\textwidth]{16_perfil_complexidade.png}
\caption{Perfil de complexidade das instâncias}
\label{fig:perfil}
\end{figure}

\section{Conclusões}

\subsection{Principais Resultados}

\begin{enumerate}
\item \textbf{Configuração mais rápida:} Teste3 (tempo médio: 32.82s)
\item \textbf{Maior taxa de sucesso:} Teste1 (98.9\%)
\item \textbf{Melhor qualidade média:} Teste2 (8.5 círculos)
\end{enumerate}

\subsection{Recomendações}

Com base nos resultados obtidos, recomenda-se:

\begin{itemize}
    \item Para instâncias pequenas (n $\leq$ 50): priorizar qualidade da solução
    \item Para instâncias médias (50 < n $\leq$ 100): balancear tempo e qualidade
    \item Para instâncias grandes (n > 100): priorizar tempo de execução
    \item Considerar timeout adaptativo baseado no tamanho da instância
\end{itemize}

\section{Arquivos Gerados}

Todos os dados e gráficos estão disponíveis no diretório \texttt{tests/analysis/results/}:

\begin{itemize}
    \item Gráficos: arquivos PNG em alta resolução (300 DPI)
    \item Dados: arquivos CSV com estatísticas detalhadas
    \item Este relatório: \texttt{relatorio\_benchmark.tex}
\end{itemize}

\end{document}
